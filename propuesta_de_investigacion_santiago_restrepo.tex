\documentclass[12pt]{article}
\usepackage{setspace}
\usepackage{helvet}

% Set Helvetica as the default font
\renewcommand{\familydefault}{\sfdefault}

% Set font size and line spacing
\fontsize{12}{18}
\setstretch{1.5}

\begin{document}

\title{Research Proposal}
\author{Waajacu: Santiago Restrepo Ruiz.}
\date{\today}



\makeatletter
\renewcommand{\maketitle}{
  \begin{titlepage}
    \begin{center}
      {\huge\bfseries \@title} % Title
      \vspace{2cm}
      
      {\large\itshape \@author} % Author
      
      \vspace{1cm}
      
      {\large \@date} % Date
      
      \vspace{2cm}
      
      % Additional item
      {\large Conflicto armado, paz negociada y posconflicto.}
      
      \vfill
    \end{center}
  \end{titlepage}
}
\makeatother

\maketitle

\section{Prefacio}
Se necesita conocer los siguientes dos conceptos ténicos centrales: \\

{ (\textbf{ZKP}) Pruebas de Conocimiento Cero (Zero-Knowledge Proofs).}\\

{ (\textbf{COOPERACION HOMOMÓRFICA}) Computación Homomórfica.}\\

\section{Sinopsis}
Se define una estrategia de cooperación entre \textit{entidades} interactivas 
en conflicto. Que trabajan en pro de un objetivo común: de naturaleza sensible, privada y/o 
de levantamiento armado con raices irresolubles. \\

La extrategia a desarrollar utiliza Pruebas de Conocimiento Cero para 
permitir a las \textit{entidads} observar a sus contrapartes sin que aquellas o estas 
se vean obligadas a revelar sus ventajas competitivas. \\

De manera simultánea, se define la Cooperación Homomórfica. \\

La Mátemática de lo que es necesario ya está resuelta, compete a la ciencia politica lo siguiente. 
\textbf{waajacu.com} se compromete a liberar con licencia libre los algoritmos necesarios, 
y a fabricar versiones utilizables en lenguaje de Ensamblador. \\

Esta investigación es importante porque busca la reducción del conflicto. 
Instar a las partes malintencionadas a comportarse de manera justa (ZKP) 
y cooperar en pro de un objetivo común (Homomorfismos).\\

\section{Pregunta de investigacion}
"¿Cómo se puede diseñar e implementar una estrategia basada en Pruebas de Conocimiento Cero 
y Computación Homomórfica para permitir la cooperación entre entidades en conflicto, 
salvaguardando la privacidad y ventajas competitivas de cada una, y promoviendo un objetivo común?"\\

\section{Planteamiento del problema}
El constante conflicto entre entidades, ya sea en el ámbito político, económico, social o militar, 
ha creado una necesidad crítica de estrategias de cooperación que permitan trabajar hacia un 
objetivo común sin comprometer la seguridad, la privacidad y las ventajas competitivas de cada parte. 
Sin embargo, existe un desafío significativo en la implementación de tales estrategias, principalmente 
en la capacidad para cooperar y verificar la cooperación de cada entidad sin revelar información.\\

\section{Objetivos}

\textbf{Objetivo General:}
\begin{itemize}
    \item{Desarrollar una estrategia basada en Pruebas de Conocimiento Cero (ZKP) y Computación Homomórfica para permitir la cooperación entre entidades en conflicto sin comprometer su privacidad ni sus ventajas competitivas, y promover la consecución de un objetivo común.}
\end{itemize}


\textbf{Objetivos Específicos:}
\begin{itemize}
    \item{Diseñar un marco teórico para una estrategia que integre el uso de ZKP y Computación Homomórfica para la cooperación en contextos conflictivos.} 
    \item{Traducir este marco teórico en algoritmos utilizables y liberarlos con licencia libre.} 
    \item{Implementar estos algoritmos en lenguaje de Ensamblador y validar su funcionamiento en situaciones de prueba.} 
\end{itemize}


\section{Lineamientos}
\textbf{Lineamientos Teóricos:}
\begin{itemize}
    \item{Pruebas de Conocimiento Cero (ZKP): Estudiaremos las teorías y aplicaciones de las ZKP, en particular cómo se pueden utilizar para verificar la información sin revelar los detalles subyacentes. Se explorará tanto la teoría matemática como las aplicaciones prácticas.}
    \item{Computación Homomórfica: Se investigará la teoría y práctica de la computación homomórfica, centrándose en cómo permite las operaciones en datos cifrados, y su relevancia para la cooperación entre entidades que desean mantener la privacidad de su información.}
    \item{Cooperación y Conflicto: Se revisarán las teorías y modelos de conflicto y cooperación, con un enfoque especial en el papel de la privacidad y las ventajas competitivas en la interacción entre entidades. También se explorará la literatura existente sobre estrategias de cooperación en situaciones de conflicto.}
\end{itemize}


\textbf{Lineamientos Metodológicos:}
\begin{itemize}
    \item{Análisis Teórico: Realizaremos un análisis detallado de las teorías y principios subyacentes a las ZKP, la computación homomórfica y la cooperación en conflictos. Esto involucrará la revisión de literatura académica y técnica relevante.}
    \item{Diseño de Algoritmos: Basándonos en nuestro análisis teórico, diseñaremos algoritmos que implementen nuestra estrategia de cooperación. Estos algoritmos serán desarrollados de manera que puedan ser liberados con licencia libre.}
    \item{Implementación en Ensamblador: Implementaremos los algoritmos diseñados en lenguaje de ensamblador. Esta implementación será realizada teniendo en cuenta la optimización y eficiencia del código.}
    \item{Análisis de Impacto: Examinaremos el potencial impacto de la estrategia en la cooperación entre entidades en conflicto, analizando la literatura existente y utilizando métodos cualitativos y cuantitativos según sea apropiado.}
\end{itemize}



\section{Bibliografía}
\begin{thebibliography}{9}
\bibitem{ZKPTheory}
    Goldwasser, Shafi, Silvio Micali, and Charles Rackoff,
    \emph{The Knowledge Complexity of Interactive Proof Systems}.
    SIAM Journal on Computing,
    vol. 18, no. 1, pp. 186-208, 1989.

\bibitem{HomomorphicComp}
    Gentry, Craig,
    \emph{A Fully Homomorphic Encryption Scheme}.
    PhD thesis, Stanford University, 2009.
    
\bibitem{AssemblyLang}
    Hyde, Randall,
    \emph{The Art of Assembly Language}.
    No Starch Press,
    2nd edition, 2010.

\end{thebibliography}

\end{document}
